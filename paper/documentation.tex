\documentclass[
12pt,
a4paper,
pdftex,
czech,
titlepage
]{article}

\usepackage[a4paper, total={6in, 9in}]{geometry}
\usepackage[czech]{babel}
\usepackage[utf8]{inputenc}
\usepackage{lmodern}
\usepackage{textcomp}
\usepackage[T1]{fontenc}
\usepackage{amsfonts}
\usepackage{titlesec}
\usepackage{float}
\usepackage{calc}  
\usepackage{hyperref}
\usepackage{enumitem}  
\usepackage{graphicx}
\usepackage{fancyvrb}
\newcommand{\latex}{\LaTeX\xspace}
\newcommand{\tex}{\TeX\xspace}

\titleformat{\chapter}
  {\normalfont\LARGE\bfseries}{\thechapter}{1em}{}
\titlespacing*{\chapter}{0pt}{0ex plus 1ex minus .2ex}{2.0ex plus .2ex}

\begin{document}

\begin{titlepage}
	\vspace*{-2cm}
	{\centering\includegraphics[scale=1.0]{logo.pdf}\par}
	\centering
	\vspace*{2cm}
	{\Large Semestrální práce z KIV/DBM2\par}
	\vspace{1.5cm}
	{\Huge\bfseries SPARQL Tool\par}
	\vspace{2cm}

	{\Large Martin Matas\par}
	{\Large A18N0095P\par}
	{\Large martinm@students.zcu.cz\par}

	\vfill

	{\Large \today}
\end{titlepage}

\tableofcontents
\thispagestyle{empty}
\clearpage

\section{Zadání}
\setcounter{page}{1}

Vytvořit Javascriptový nástroj, který bude snadno vložitelný do jiných
webových projektů (ideálně jeden javascript soubor k načtení v html
hlavičce) a bude umět:

\begin{itemize}
\item
  odeslat SPARQL dotaz na zadaný libovolný SPARQL endpoint (ajax)
\item
  vypsat výsledek dotazu do triviální webové tabulky
\item
  načíst/uložit dotaz do místní paměti problížeče (browser local
  storage)
\item
  spravovat uložené dotazy
\item
  vyhledávat v uložených dotazech
\item
  verzovat dotazy
\end{itemize}

\hypertarget{data}{%
\subsection{Data:}\label{data}}

\begin{itemize}
\item
  dle svého uvážení, případně na vyžádání dodám SPARQL datazy a
  příslušné datasety
\end{itemize}

\section{Řešení}

Nástroj byl po zvážení vypracován pomocí jazyka \texttt{TypeScript}, který umožňuje přeložení zdrojového kódu do Javascriptu ve standardu ECMAScript 5 (ES5). Standard ES5 byl zvolen předevší pro svoji rozšířenost, protože v dnešní době je podporován většinou prohlížečů a knihoven. Další nespornou výhodou jazyka TypeScript byla možnost napsat nástroj s využitím typové kontroly a možností OOP programování, díky čemuž je kód snadno pochopitelný a přehledný. 

Pro správu závislostí a sestavení projektu byl využit nástroj \texttt{npm}.

\subsection{Struktura aplikace}
\label{sec:structure}

Popis jednotlivých tříd a jejich využití je podrobně rozepsáno v přiložené dokumentaci v adresáři \texttt{docs/}.

\subsection{Struktura úložiště}

Aby bylo možné efektivně pracovat s dotazy uloženými v lokálním úložišti prohlížeče (localstorage), byla vytvořena jednoduchá interní struktura viz zdrojový kód níže. 

\begin{Verbatim}[samepage=true]
{
  queries: [						
    {
      name: string,
      queryString: Array<string>,
      currentVersion: number
    }
  ]
}
\end{Verbatim}

Každý dotazy typu \texttt{Query} je uložen uvnitř pole \texttt{queries}. Aby bylo možné dotazy verzovat, každý dotaz obsahuje pole všech verzí dotazu \texttt{queryString} a ukazatel definující aktuální verzi \texttt{currentVersion}.

\section{Uživatelská dokumentace}

\subsection{Sestavení}

Nástroj obsahuje již přeložené zdrojové kódy a vygenerovanou dokumentaci, ale v případě potřeby (např. při změně zdrojového kódu knihovny) lze projekt sestavit pomocí nástroje \texttt{npm}, který má definováno několik skriptů:

\begin{description}
    \item[clean] - vyčištění projektu od přeložených souborů a generované dokumentace
    \item[tsc] - přeložení knihovny
    \item[minify] - minifikace a znečitelnění zrojových souborů
    \item[docs] - vygenerování dokumentace kódu
    \item[build] - sestavení projektu
    \item[build:min] - sestavení projektu + minifikace a znečitelnění zrojových souborů
\end{description}

\noindent Jednotlivé skripty se spustí příkazem:

\begin{verbatim}
... \> npm run název_skriptu
\end{verbatim}

\subsection{Práce s knihovnou}

Popis práce s knihovnou a klíčovými třídami je podrobně popsán v souboru \texttt{README.md}.

\section{Závěr}

Nástroj splňuje požadavky zadání a je kompatibilní se všemi prohlížeči. Při implementaci jsem nenarazil na žádný problém. Problémy nastaly až při sestavení projektu, kdy nebylo jednoduché nastavit konfigurace tak, aby se generoval jeden soběstačný JS soubor.

\end{document}